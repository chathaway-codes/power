%% Template file for all Software/Hardware modules

% Replace "Name of Module" with the name of this module
\subsection{Factory Reset}

\subsubsection{Description}

This module describes the architecture of how the backend will manage features related to the factory reset.
The factory reset will restore everything to the exact same way it was when the system came out of the "factory".
This means that all data will be wiped, and custom modifications will be wiped, and any software upgrade will be wiped.

\subsubsection{Program Flow}

This function will be invoked either through a call passed down from the web interface (it will be handled by the \ac{REST} \ac{API}, and the call will be intercepted and result in an action) or by a program picking up on the push of the physical factory-reset button.

\subsubsection{Data Flow}

Doing the factory restore will involve a good bit of trickery. 
First, we will need to put a partition on the disk that stores all the files that make up the "live" running system. 
This might be best done by simply putting a tarball and a script on a small partition at the end of the disk. 
The script will be responsible for wiping the primary partition, re-creating it, and extracting the tarball.

Getting the system to boot into the "restore" mode will involve modifying grub settings on the fly. 
We can do this, and it shouldn't be too hard, but it feels like a really bad idea.

After the machine is in recovery mode, something will need to launch the script to do the restore.
This shouldn't be too difficult if just use an old-school \filename{rc.conf} script or something of the sort.
The script will also restore grub, so we don't have to worry about returning that to it's old good state.

\subsubsection{Potential Problems}

One of the biggest risks here is failing to boot into the restore partition and ruining our grub configuration.
This would cause the customer to need to send us back the unit, we would have to flash it, and send it back.
If we can get this feature working however, we will not have to worry so much about breaking the web UI by accident.
