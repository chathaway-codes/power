
\section{Definitions}

\paragraph{Amperes (Amps)}
The SI base unit of electric current, equal to a flow of one coulomb per second.

\paragraph{API}
An application programming interface (API) is a specification intended to be used as an interface by software components to communicate with each other.

\paragraph{AJAX}
AJAX is a way of executing asynchronous calls to the server by using Javascript. 
The benefit of this is that we don't need to refresh the page, or send extra data over the network.

\paragraph{HTTP}
The Hypertext Transfer Protocol (HTTP) is an application protocol for distributed, collaborative, hypermedia information systems. 
HTTP is the foundation of data communication for the World Wide Web.

\paragraph{IEEE 802.15.4}
A low-power, low-data rate wireless protocol for \ac{PAN}s.

\paragraph{JSON}
Javsascript Object Notation. 
JSON is a way of representing objects in a hash-table that is easy to use in the Javascript scripting language. 
This is basically a hashmap, and can be looped through using for-each loops, for loops, and while loops.

\paragraph{REST}
REpresentational State Transfer is a software architecture that makes use of the existing HTTP protocol to implement a well defined, practical, and efficient means of representing resources on the web.
REST uses URL's to denote different resources, with the query string being used to filter the resources. 
Although some believe that using a "clean url" instead of the query string is proper REST, for our purposes there is no reason to have especially "human-readable" URL's that result from using clean URL's.
In short, this is how we would denote a device in the REST API:

/api/raw/device/5

/api/raw is simple the entry path to the API. 
The next part, /device, is the table we are accessing. 
The last part, /5, is the primary key of the object we are looking at.

Another way to represent the same device, using the query string, would be:

/api/raw/device/?id=5

If you simple open these URL's in a web browser, you will be initiating a GET request. A GET request returns data already on the server. The verbs in the HTTP vocabulary are:

\begin{itemize}
 \item GET - Get's information about the device
 \item POST - Creates a new entry, if directed at the resource collection
 \item PUT - Updates the specified resource with the given information
 \item DELETE - Deletes the specified resource or everything in the specified collection
\end{itemize}

Combing these verbs with the search query and the URL, we can effectively create an uniform API for accessing the database over the web.
In addition, we can control access and authorization via standard methods handled by the web.

\paragraph{Volt (Volts)}
The SI unit of electromotive force, the difference of potential that would carry one ampere of current against one ohm resistance.

\paragraph{Watt (Watts)}
The SI unit of energy transfer, defined as one joule per second. Watts, denoted by P, can be calculated by multiplying the Volts and Amps of a given system.

\paragraph{XBee}
XBee is the name given to small radio modules made by Digi. The XBee radios used in this project are XBee Series 2 radios, which are ZigBee compliant.

\paragraph{ZigBee}
ZigBee is a standard that uses IEEE 802.15.4 as it's base. XBee Series 2 radios use ZigBee to form an ad-hoc mesh network that can add or remove nodes at any time.