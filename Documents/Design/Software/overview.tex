%% Template file for all Software/Hardware modules

% Replace "Name of Module" with the name of this module
\chapter{Software Design Overview}

\section{Description}

The POWER project uses various pieces of software to control the Satellites, provide an API for displays, and generate pages for the primary display. The POWER project will consist of the following pieces of software:
\begin{itemize}
 \item The microprocessor code
 \item The server code to communicate with the Satellites
 \item The server code to present the API
 \item The client code to render the display for the user
\end{itemize}

The microprocessor code is documented in the hardware section because it is very related to the hardware, and it is much more logical to put the  code there.

\subsection{The Display Backend}

The general plan for the backend is to write modules for a Django web application. These modules will either be running in the background and reporting to the through the REST API functions, or responding to user requests.

The logic for using Django as the base, and write modules around Django, is one of necessity. We do not have enough time or manpower to make everything we need from scratch. Therefore, we will be using several open source utilities to assist us. 

Django will give us an easy-to-use database interface. It will deal with sanitizing all SQL queries, and make sure all output is escaped (ie, no XSS). Using Django gives us almost all of the security benefits listed in our non-functional requirements.

To make our REST API, we will utilize an open-source Django library called tastypie. Tastypie gives us both model resources (that is, api elements that are basically the database tables) and the ability to create custom resources that may, or may not, map to the database. The one (that's right, one) instance where we may not want to directly map something to the database is when we are looking at turning the outlets on and off.

Another advantage of Django is that is comes with a very extensive, expandable, and adaptable authentication system. This will help accomplish several optional and required features almost instantly. 

\subsection{The Display Frontend}

The general plan for the frontend is to serve up some fairly static pages, and uses extensive amounts of Javascript to fill in the content and animate things. The frontend will communicate with the backend through the REST API. This will be fairly simply, since the backend API will support multiple formats, including JSON.

\section{Program Flow}

This program will have three main "threads". One thread would be responsible for listening to messages from the Satellites, and another would be responsible for listening and responding to requests on the API. The last thread would be responsible for serving up static files for the main display.

The Satellite thread will be responsible for taking the logging information provided by the Satellites and storing it in the database. Each Satellite will report it's information to the server once every second. Therefore, all this thread needs to do is react to incoming connections.

The API thread would provide the Representation State Transfer API (REST) that will be used by all future displays to get data in a standardized, well defined way. This thread will NOT be serving up images or page templates. This is done to make best use of the caching functionality included in the HTTP server software that will be used.

The last thread will be running a "dumb" web page that servers up static HTML and Javascript files. This thread will also retrieve images and CSS files from the server.

\section{Data Flow}

This application will use a SQL database to store all data. Each module will interact with the REST API, which interacts with the database as needed. The diagram for this database is broken down in the "Sub-modules" section, with a diagram for each logical grouping of tables.

\begin{figure}
\centering
\includegraphics[scale=0.75]{Software/images/DataFlowDiagram.png}
\caption{Data Flow Diagram}
\label{DataFlowDiagram}
\end{figure}

Figure \ref{DataFlowDiagram} shows how various components will be interacting. Notice that the database is not directly accessed by any components, but goes through the REST API instead. This allows us to provide a standard interface should be more-or-less accessible from any programming language or system. This will also allow us to use some well-known security features, such as TLS.

\section{UML Diagrams}

% Any diagrams that can describe the system design
%  Such as inheritance and actors

\section{Potential Problems}

Using open-source libraries requires us to learn them. This is a big problem because of the time constraints on this project, and will require us to spend almost as much time learning as we will be in development (possibly more). This can lead to frustration, especially in light of the fact that we will need to learn the languages in addition to the libraries.

However, using well developed and designed libraries allows us to skip over a good part of the development and planning. We will be able to use extensive API's without taking responsibility for maintaining the software. This will allow us to focus on our product instead of the all the small things that make it up. 

\section{The Backend}

% The backend things
%% Template file for all Software/Hardware modules

% Replace "Name of Module" with the name of this module
\subsection{Authentication}

\subsubsection{Description}

This module describes the architecture of how the backend will manage features related to users and groups (authentication features).

The backend authentication management provided by Django is very robust. It includes user authentication, access control (including groups and individual permissions), administrative panels, and logging of who does what. 

\subsubsection{Program Flow}

The authentication module will be called whenever a user tries to access something. It will verify that the user is logged in, has permissions, and even write a short log message if it is an administrative function.

\subsubsection{Data Flow}

Because this module is not being written by us, it will not be following the standards we are using. Therefore, it will be using the standard Django classes to access database objects instead of going through the REST API. 

While that is not ideal, it should be OK anyway. This part of the Django framework has been tested extensively, in many production environments. There is a low risk that something will not work, and if there is a problem there is a very high chance that upon being reported to the Django project, it will be fixed quickly.

\subsubsection{Potential Problems}

% A list of potentional programs along with suggestions
%  on ways to work around them. Elaborate on why the problem
%  exists

\subsubsection{Sub-modules 1}

% This is a second subsubsection of modules,
%  and should consists of this module broken 
%  down further into components
%% Template file for all Software/Hardware modules

% Replace "Name of Module" with the name of this module
\subsection{The REST API}

\subsubsection{Description}

This module describes the architecture of how the backend will manage features related to providing access to the REST API.

The REST API will provide a urls.py file, which will provide a handler for all REST methods.
These could be achieved by simply using the tastypie URL's, which can be generated by calling include(api\_v1.urls) in the urlpatterns object.

\subsubsection{Program Flow}

The REST API will be activated by the Django chain when a request comes in for the access to the API, or when another module directly calls functions in the API during their response to a request from the user.

\subsubsection{Data Flow}

All data will be returned to the calling client either as a Python dictionary (if called directly from another module), or as JSON, XML, or YAML through a web request (as specified in the query filter using ?format=json).

The REST API will directly access the database.
Initially, tastypie will be dealing with converting the HTTP request to a method on the Django objects.
Django will be generating the SQL that actually gets run, and encapsulating objects in Python classes.


\subsubsection{Potential Problems}

At first, we will be using a prepackaged Django module to help provide the REST API.
This could be problematic because we don't know a lot about the implementation of the API, and if there is a problem tracking it down could be difficult.

The good news is that django-tastypie is well supported by the community, and source code is liberally commented.
It follows Python commenting standards, and should be fairly easy to navigate.
In addition, members of our team have already used tastypie for projects in both work and the garage.


The most problematic bit about using tastypie for the initial run-through is the compatibility problems it has with Backbone.js.
However, the community has developed an addition to Backbone to help deal with this.
It is called Backbone-tastypie.
It is a simple modification to the Backbone sync operations that allows it to communicate with a standard tastypie REST-API.

%% Template file for all Software/Hardware modules

% Replace "Name of Module" with the name of this module
\subsection{Satellite}

\subsubsection{Description}

This module describes the architecture of how the back end will manage features related to managing Satellites.

The Satellite Management back end is in control of maintaining the information in the database. 


\subsubsection{Program Flow}

% Insert a flow chart here with a high-level look
%  at the states this part of the system goes through

\subsubsection{Data Flow}
The back end Satellite Management module will receive data from the Satellite Management front end via the REST API. 
This includes data that needs to be changed in the database, and requests for data. 
This module will then modify or query the database as needed, 
returning the requested data or a confirmation of modification to the front end through the REST API. 

% Describe where data goes and comes from in this subsubsection
% Flow charts are encouraged

\subsubsection{UML Diagrams}

% Any diagrams that can describe the system design
%  Such as inheritance and actors

\subsubsection{Potential Problems}

% A list of potentional programs along with suggestions
%  on ways to work around them.
% Elaborate on why the problem exists

\subsubsection{Sub-modules 1}

% This is a second subsubsection of modules,
%  and should consists of this module broken 
%  down further into components
%% Template file for all Software/Hardware modules

% Replace "Name of Module" with the name of this module
\subsection{Devices}

\subsubsection{Description}

This module describes the architecture of how the back end will manage features related to Device Management. 

The Device Management back end will be in charge of maintaining the database with information pertaining to the Devices. 
The back end will also generate the framework for the web pages. 
It will send the basic format of the Device Management page to the front end to be filled in with additional formatting. 
 

\subsubsection{Program Flow}

The back end Device Management module will be activated by a request from the front end Device Management. 
This module will complete the request by adding information to the SQL database or querying said database for specific data. 

\subsubsection{Data Flow}

The Data Flow will behave in much the same manner as the Program Flow. 
The back end Device Management module will receive data from the Device Management front end via the REST API. 
This includes data that needs to be changed in the database, and requests for data. 
This module will then modify or query the database as needed, 
returning the requested data or a confirmation of modification to the front end through the REST API. 
The data for the basic format of the web pages will also be sent from the back end to the front end through the REST API. 

%\subsubsection{Potential Problems}

% A list of potentional programs along with suggestions
%  on ways to work around them.
% Elaborate on why the problem exists
%% Template file for all Software/Hardware modules

% Replace "Name of Module" with the name of this module
\subsection{Power Bill Guestimator}

\subsubsection{Description}

The Power Bill Guestimator backend will be in charge of maintaing a record of utility power costs for different times and usage rates.
These rates are set by the user on the front end and stored in the database.
It will be able to calculate the cost of the power used between certain time ranges and give that data to the front end to be formatted.

\subsubsection{Program Flow}

Data flows through this module when the front end requests the cost of power used for Satelites. 
This triggers a query of data from the database of the power usage history for that Satelite (or combination of Satelites) and a query of the rates that applied to the time period.
With that data the module will calculate how much power was used at each rate and give that data back to the front end.


\subsubsection{Potential Problems}

Since rates can vary based on time of day, total power used, or other unknown factors, keeping track of rates may be tricky. Research must be done to find out the most common ways utilities charge for power and be able to store rates accordingly.

%% Template file for all Software/Hardware modules

% Replace "Name of Module" with the name of this module
\subsection{Factory Reset}

\subsubsection{Description}

This module describes the architecture of how the backend will manage features related to the factory reset.

\subsubsection{Program Flow}

% Insert a flow chart here with a high-level look
%  at the states this part of the system goes through

\subsubsection{Data Flow}

% Describe where data goes and comes from in this subsubsection
% Flow charts are encouraged

\subsubsection{UML Diagrams}

% Any diagrams that can describe the system design
%  Such as inheritance and actors

\subsubsection{Potential Problems}

% A list of potentional programs along with suggestions
%  on ways to work around them. Elaborate on why the problem
%  exists

\subsubsection{Sub-modules 1}

% This is a second subsubsection of modules,
%  and should consists of this module broken 
%  down further into components

\section{The Frontend}

% And these are the frontend things
%% Template file for all Software/Hardware modules

% Replace "Name of Module" with the name of this module
\subsection{Authentication}

\subsubsection{Description}

This module describes the architecture of how the frontend will create pages that allow the user to manage user and groups.

The authentication front end will consist of both the "login" page, and the user management pages.
These pages either post directly to the server using standard form-submission methods (ie, not \ac{AJAX}) or by using \ac{AJAX} (if we have time to recreate the entire management system).

\subsubsection{Program Flow}

The user will first be presented with a login screen, which will take two pieces of information:
\begin{itemize}
 \item Username - The user's username
 \item Password - The user's password
\end{itemize}

After these two pieces of information have been verified (ie, that password belongs to that user), the user will be logged in.

From here, the user has all kinds of options.
The only ones that are interesting to this module is the settings panel, which includes a link to the user administration panel.
The user administration panel will, initially, be a themed Django-admin view.
Time permitting, this may be recreated to be a \ac{AJAX} type interface.

\subsubsection{Data Flow}

All data will be sent to the server through \ac{POST} requests, not through the \ac{REST} \ac{API}.
This will allow us to skip over this module almost entirely.
This is not consistent with everything else, and will need to be remade later on if there is time.

\subsubsection{Potential Problems}

Themeing the admin interface to match our site will be tricky.
At best, it will involve telling Django to include some extra stylesheets or using a different template.
At worst, it will involve modifying the Django stylesheets directly.
Either option works, but due to version control it would greatly preferred to be able to tell Django what styles to use.

%% Template file for all Software/Hardware modules

% Replace "Name of Module" with the name of this module
\subsection{Satellite}

\subsubsection{Description}

This module describes the architecture of how the frontend will create pages that allow the user to add, remove, and modify Satellites associated with the system.

\subsubsection{Program Flow}

The Satellite Management front end consists of the Satellite Management page. 
This page leads to a set-up wizard for syncing new Satellites with the Server. 
The wizard walks the user through instructions 

% Insert a flow chart here with a high-level look
%  at the states this part of the system goes through

\subsubsection{Data Flow}

% Describe where data goes and comes from in this subsubsection
% Flow charts are encouraged

\subsubsection{UML Diagrams}

% Any diagrams that can describe the system design
%  Such as inheritance and actors

\subsubsection{Potential Problems}

% A list of potentional programs along with suggestions
%  on ways to work around them.
Elaborate on why the problem
%  exists

\subsubsection{Sub-modules 1}

% This is a second subsubsection of modules,
%  and should consists of this module broken 
%  down further into components
%% Template file for all Software/Hardware modules

% Replace "Name of Module" with the name of this module
\subsection{Devices}

\subsubsection{Description}

This module describes the architecture of how the frontend will create pages that allow the user to manage devices in the system. 

It will receive a basic framework for the web pages in HTML from the back end module and use Javascript to render the rest of the page. 
The pages will be created with a mix of HTML and CSS from the back end and heavy Javascript from this module. 
The Device Management front end includes the Device Management page, Add Devices page, and the Confirm Disable Device dialog. 


\subsubsection{Program Flow}

The user will be presented with the option to add, edit, or disable a Device. 
Adding a Device allows the user to name the new Device and associate it with a specific Satellite. 
Editing a Device allows the user to change the Satellite associated with the Device. 
The user will not have the option to edit the name of the Device once it has been created in order to maintain consistant data. 
Disabling a Device will remove it from all graphs and tables and data will no longer be collected on the Device.
Devices cannot be deleted in order to maintain past data.

\subsubsection{Data Flow}

The data will be sent to the Device Management back end module through the REST API. 
The back end returns either a confirmation of action or the requested data to the front end via the API. 
This module receives the basic format for the web pages from the back end Device module. 
It then fills in the additional and page-specific content. 

%\subsubsection{Potential Problems}

% A list of potentional programs along with suggestions
%  on ways to work around them.
%  Elaborate on why the problem exists

%% Template file for all Software/Hardware modules

% Replace "Name of Module" with the name of this module
\subsection{Frontend - View Data}

\subsubsection{Description}

This module describes the architecture of how the frontend will create pages that allow the user to view various pieces of data in various formats.

The View Data front end is responsible for providing the user with data regarding each device and device group. 
It will provide three separate graphs and tables with data specified by the user; 
Device Power Consumption, Monthly Power Consumption, and Power Consumption Over Time. 


\subsubsection{Program Flow}

From the Home page the user will be able to select which graph or table they would like to view.  
These graphs and tables provide the user with information pertaining to the Devices specified and within the time-frame selected.  

\subsubsection{Data Flow}

This module requests data for the graphs and tables from the back end through the REST API. 


% Describe where data goes and comes from in this subsubsection
% Flow charts are encouraged

%\subsubsection{UML Diagrams}

% Any diagrams that can describe the system design
%  Such as inheritance and actors

\subsubsection{Potential Problems}

% A list of potentional programs along with suggestions
%  on ways to work around them.
% Elaborate on why the problem exists

%\subsubsection{Sub-modules 1}

% This is a second subsubsection of modules,
%  and should consists of this module broken 
%  down further into components
%% Template file for all Software/Hardware modules

% Replace "Name of Module" with the name of this module
\subsection{Power Bill Guestimator}

\subsubsection{Description}

This module describes the architecture of how the frontend will create pages that allow the user to view guestimates of their monthly power bill, in various scenarios.

It will receive a basic framework for the web pages in \ac{HTML} from the back end module and use Javascript to render the rest of the page. 
The pages will be created with a mix of \ac{HTML} and \ac{CSS} from the back end and heavy Javascript from this module. 

\subsubsection{Program Flow}

When the user would like to estimate a power bill they first have to select what devices they want to view and for what time period.
With that information selected, this module asks the backend for the power cost data for said devices.

When the user is modifiying the utility company rates they will be presented with a wizard that ill guide them through inputting them.
This wizard will ask them several questions about how the utility company charges for power and will give data to the backend accordingly.


\subsubsection{Potential Problems}

One potential problem is having a utility company have rates change in a way we did not anticipate.
One other issue is not having rates defined before trying to calculate the bill.

%% Template file for all Software/Hardware modules

% Replace "Name of Module" with the name of this module
\subsection{Factory Reset}

\subsubsection{Description}

This module describes the architecture of how the frontend will create pages that allow the user to reset the system back to factory default settings.
It will receive a basic framework for the web pages in HTML from the back end module and use Javascript to render the rest of the page. 
This front end module will be a page that will allow the user to initiate a Factory Reset.

\subsubsection{Program Flow}

When the user loads this page they will be presented with an option to perform a factory reset.
If they choose that option, this module will ask the user if they are sure and list the consiquences of doing so.
If the user still agrees, it will request a factory reset to be performed from the back end module.

\subsubsection{Potential Problems}

Making sure only administrators can reset the device to reduce the risk of accidentally breaking it.
Also may want to implement a method of backing up and restoring all of the data.


% This is a second section of modules,
%  and should consists of this module broken 
%  down further into components

