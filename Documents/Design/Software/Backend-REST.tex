%% Template file for all Software/Hardware modules

% Replace "Name of Module" with the name of this module
\subsection{The REST API}

\subsubsection{Description}

This module describes the architecture of how the backend will manage features related to providing access to the REST API.

The REST API will provide a \filename{urls.py} file, which will provide a handler for all REST methods.
These could be achieved by simply using the tastypie URL's, which can be generated by calling \filename{include(api\_v1.urls)} in the urlpatterns object.

\subsubsection{Program Flow}

The REST API will be activated by the Django chain when a request comes in for the access to the API, or when another module directly calls functions in the API during their response to a request from the user.

Having each module register itself when the system launches should work, but there may be a problem with things getting registered twice.


\subsubsection{Data Flow}

All data will be returned to the calling client either as a Python dictionary (if called directly from another module), or as JSON, XML, or YAML through a web request (as specified in the query filter using ?format=json).

The REST API will directly access the database.
Initially, tastypie will be dealing with converting the HTTP request to a method on the Django objects.
Django will be generating the SQL that actually gets run, and encapsulating objects in Python classes.


\subsubsection{Potential Problems}

At first, we will be using a prepackaged Django module to help provide the REST API.
This could be problematic because we don't know a lot about the implementation of the API, and if there is a problem, tracking it down could be difficult.

The good news is that django-tastypie is well supported by the community, and source code is liberally commented.
It follows Python commenting standards, and should be fairly easy to navigate.
In addition, members of our team have already used tastypie for projects in both work and recreation.

The most problematic bit about using tastypie for the initial run-through is the compatibility problems it has with Backbone.js.
However, the community has developed an addition to Backbone to help deal with this.
It is called Backbone-tastypie.
It is a simple modification to the Backbone sync operations that allows it to communicate with a standard tastypie REST API.

When the web server is launched, it initialized all modules by calling their \filename{\_\_init\_\_.py} file.
In this file, each module can register themselves with the API.
This will allow all modules to be self-contained, which is ideal in that it will allow us to upgrade one module without breaking everything else.

Early experiments indicate that the \filename{\_\_init\_\_.py} file is called twice during the loading of the system.
This could be solved a number of ways, or it might not even be a problem. 
It's very possible the first time the module is loaded it is just being checked for syntax and compiled, then the program is restarting with the compiled files.
